\chapter{Конструкторский раздел}

\section{Описание структур данных}

Для формализации общего алгоритма синтеза изображения в данной программе, необходимо ввести определения использующихся в ней структур данных:
\begin{itemize}
	\item Сцена представляет собой список с произвольным числом моделей и объект камеры.
	\item Модель включает в себя следующие данные: \begin{itemize}
		\item массив вершин фигуры;
		\item массив рёбер фигуры;
		\item массив полигонов фигуры;
		\item цвет поверхности.
	\end{itemize}
	\item Камера содержит: \begin{itemize}
		\item положение в пространстве;
		\item систему координат камеры, задаваемую тремя ортогональными векторами;
	\end{itemize}
\end{itemize}

Вершины задаются координатами $(x, y, z)$, ребро — двумя индексами вершин в массиве вершин, полигон — тремя индексами вершин в массиве вершин.
Для последующей растеризации более удобный формат хранения полигона — треугольник, заданный сразу тремя вершинами.

\section{Моделирование поверхности функции от двух переменных}

Чтобы перейти от рассмотрения функции к рассмотрению поверхности, необходимо построить модель, то есть представить объект в виде полигонов, аппроксимирующих поверхность, заданную функцией.
В данной программе поверхности ограничены по двум координатам, соответственно, можно составить ортогональную сетку, в узлах которой будут значения функции.
Так как ортогональная сетка представляется прямоугольниками, необходимо каждую ячейку сетки разделить на треугольники.
Триангуляцию можно осуществить двумя путями: соединить вершины диагонали прямоугольника или выделить новую вершину в центре прямоугольника, соединив её со всеми вершинами этого прямоугольника.
В данной программе выбран второй способ (рис. \ref{img:grid}).

\img{60mm}{grid}{Сетка поверхности}

\section{Аффинные преобразования}

Для реализации возможности поворота и масштабирования относительно наблюдаемых поверхностей, необходимо воспользоваться матрицами аффинных преобразований \cite{Newman}.

Матрица масштабирования относительно начала координат с коэффициентами $k_x$, $k_y$, $k_z$:
\begin{equation}
	\begin{pmatrix}
		\frac{1}{k_x} & 0 & 0 & 0 \\
		0 & \frac{1}{k_y} & 0 & 0 \\
		0 & 0 & \frac{1}{k_z} & 0 \\
		0 & 0 & 0 & 1 \\
	\end{pmatrix}
\end{equation}

Матрицы поворота на угол $\varphi$
\begin{itemize}
	\item относительно оси $oX$:
		\begin{equation}
			\begin{pmatrix}
			1 & 0 & 0 & 0 \\
			0 & \cos\varphi & \sin\varphi & 0 \\
			0 & -\sin\varphi & \cos\varphi & 0 \\
			0 & 0 & 0 & 1 \\
		\end{pmatrix},
	\end{equation}
	
	\item относительно оси $oY$:
	\begin{equation}
		\begin{pmatrix}
			\cos\varphi & 0 & \sin\varphi & 0 \\
			0 & 1 & 0 & 0 \\
			-\sin\varphi & 0 & \cos\varphi & 0 \\
			0 & 0 & 0 & 1 \\
		\end{pmatrix},
	\end{equation}
	
	\item относительно оси $oZ$:
	\begin{equation}
		\begin{pmatrix}
			\cos\varphi & \sin\varphi & 0 & 0 \\
			-\sin\varphi & \cos\varphi & 0 & 0 \\
			0 & 0 & 1 & 0 \\
			0 & 0 & 0 & 1 \\
		\end{pmatrix}.
	\end{equation}
\end{itemize}

Матрица перенос вдоль координатных осей на $dx$, $dy$, $dz$:
\begin{equation}
	\begin{pmatrix}
		1 & 0 & 0 & 0 \\
		0 & 1 & 0 & 0 \\
		0 & 0 & 1 & 0 \\
		dx & dy & dz & 1 \\
	\end{pmatrix}.
\end{equation}

\section{Разработка алгоритма, объединяющего Z-буфер и закраску по Гуро}

В аналитическом разделе был предложен метод сокращения объемов вычисления за счет исключения этапа формирования промежуточного массива точек.
Схема предложенного алгоритма показана на рисунке \ref{img:zg}.

\imgext{height=230mm}{zg}{pdf}{Объединение алгоритма z-буфера и алгоритма закраски Гуро}

\section*{Вывод}

Были описаны структуры данных, способ моделирования поверхности, геометрические преобразования камеры, схема алгоритма синтеза изображения.
