\chapter*{Введение}
\addcontentsline{toc}{chapter}{Введение}

В наши дни в компьютерной графике значительное внимание уделяется алгоритмам построения реалистических изображений.
Данные алгоритмы являются крайне затратными по времени, поскольку требуется обрабатывать модели, состоящие из огромного количества геометрических примитивов.
Они должны предусматривать множество физических явлений, таких как преломление, отражение, рассеивание света.
Благодаря росту вычислительной производительности современных компьютеров синтез изображений всё чаще находит применение в системах автоматизированного проектирования, дизайне, системах виртуальной реальности, научной визуализации, медицине, индустрии электронных развлечений, кинематографе и мультипликации.

Целью данного курсового проекта является разработка программы визуализации поверхностей, заданных функциями от двух переменных.

\section*{Задачи работы}

В рамках выполнения работы необходимо решить следующие задачи:
\begin{itemize}
	\item Изучение и анализ существующих алгоритмов компьютерной графики, которые используются для создания реалистичной модели взаимно перекрывающихся объектов, и выбор наиболее подходящих для решения поставленной задачи.
	\item Проектирования архитектуры программы и её интерфейса.
	\item Реализация выбранных алгоритмов и структур данных.
	\item Проведение исследования на основе разработанной программы.
\end{itemize}
