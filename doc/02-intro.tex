\chapter*{Введение}
\addcontentsline{toc}{chapter}{Введение}

% ПОМЕНЯТЬ ВВЕДЕНИЕ
Компьютерная графика – это раздел программирования, предметом изучения которого являются методы создания реалистичных изображений.
Алгоритмы, лежащие в основе этих методов, являются крайне затратными по времени, поскольку требуется обрабатывать модели, состоящие из огромного количества геометрических примитивов.
Благодаря росту вычислительной производительности современных компьютеров синтез изображений всё чаще находит применение в системах автоматизированного проектирования, дизайне, системах виртуальной реальности, научной визуализации, медицине, индустрии электронных развлечений, кинематографе и мультипликации.

Одной из фундаментальных задач компьютерной графики является визуализация поверхностей.
Решения этой задачи находят широкое применение при моделировании различных деталей и конструкций, для пакетов программ математической обработки.

Целью данного курсового проекта является разработка программы визуализации поверхностей, заданных функциями от двух переменных.

\section*{Задачи работы}

В рамках выполнения работы необходимо решить следующие задачи:
\begin{itemize}
	\item изучение и анализ существующих алгоритмов компьютерной графики, которые используются для создания реалистичной модели взаимно перекрывающихся объектов, и выбор наиболее подходящих для решения поставленной задачи;
	\item проектирования архитектуры программы и её интерфейса;
	\item реализация выбранных алгоритмов и структур данных;
	\item проведение исследования на основе разработанной программы.
\end{itemize}
