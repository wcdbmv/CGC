\chapter{Исследовательский раздел}

\section{Технические характеристики}

Тестирование  производительности  программы  производилось  на компьютере со следующими техническими характеристиками:
\begin{itemize}
	\item операционная система: Ubuntu 19.10 64-bit;
	\item память: 3,8 GiB;
	\item процессор: Intel® Core™ i3-6006U CPU @ 2.00GHz (4 логических ядра).
\end{itemize}

\section{Зависимость производительности программы от количества полигонов каркасных моделей}

Целью эксперимента является сравнение времени генерации изображения от количества полигонов, аппроксимирующих поверхности.
Замеры производились для сцены $720\times720$ на одном потоке.
В ходе эксперимента количество полигонов менялось от 100 до 1000 с шагом 100.
Результаты представлены на рисунке \ref{plt:polygons}.

\newpage

\begin{figure}[!h]
	\centering
	\begin{tikzpicture}[scale=0.9]
		\begin{axis}[
			axis lines=left,
			xlabel={Количество полигонов},
			ylabel={Время, мс},
			ymin=0,
			ymax=50,
			legend pos=north west,
			ymajorgrids=true
		]
		\addplot table[x=polygons,y=time,col sep=comma]{inc/csv/polygons.csv};
		\end{axis}
	\end{tikzpicture}
	\captionsetup{justification=centering}
	\caption{Зависимость времени генерации изображения от количества полигонов, аппроксимирующих поверхности функций}
	\label{plt:polygons}
\end{figure}


\section{Зависимость производительности программы от количества потоков}

Целью эксперимента является сравнение времени генерации изображения от количества потоков.
Замеры производились для сцены разрешением $600\times600$ пикселей с 500 полигонами.
В ходе эксперимента количество потоков менялось от 1 до 64, на каждом шаге их число удваивалось.
Результаты представлены на рисунке \ref{plt:threads}.

\begin{figure}[h!]
	\centering
	\begin{tikzpicture}[scale=0.9]
		\begin{axis}[
			axis lines=left,
			xlabel=Количество потоков,
			ylabel={Время, с},
			ymin=0,
			ymax=0.025,
			legend pos=north west,
			ymajorgrids=true
		]
		\addplot table[x=threads,y=time,col sep=comma]{inc/csv/threads.csv};
		\end{axis}
	\end{tikzpicture}
	\captionsetup{justification=centering}
	\caption{Зависимость времени генерации изображения от количества потоков}
	\label{plt:threads}
\end{figure}

\section*{Вывод}
Как видно из результатов экспериментов, максимальной скорости работы удалось добиться при 4 потоках (быстрее, чем для 1 потока на 55\%), что равно количеству логических ядер компьютера, на котором проводилось тестирование. 
Время генерации изображения линейно зависит от количество полигонов на сцене.
