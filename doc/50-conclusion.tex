\chapter*{Заключение}
\addcontentsline{toc}{chapter}{Заключение}

%% ПЕРЕДЕЛАТЬ

В результате проделанной работы был разработан программный продукт, позволяющий визуализировать поверхности функций от двух переменных..

В процессе разработки были рассмотрены, проанализированы и реализованы основные алгоритмы построения реалистичного трехмерного изображения: алгоритмы удаления невидимых линий и методы закраски.

Разработанный программный продукт предоставляет возможности настройки визуальных характеристик поверхностей и положения камеры.
Были добавлены возможности изменения ориентации и положения камеры с помощью манипуляторов клавиатура и мышь.

Данный программный продукт может быть использован для демонстрации сложных графиков функций от двух переменных, их взаимного расположения.

В исследовательском разделе расчетно-пояснительной записки приведена оценка производительности программы при различном количестве граней, аппроксимирующих поверхности функций.

Была предложена оптимизация алгоритма генерации изображения путем совмещения алгоритма z-буфера и алгоритма закраски по Гуро, что позволило строить реалистичное трехмерное изображение поверхностей функций в реальном масштабе времени.

Благодаря использованию объектно-ориентированного подхода при разработке, в программный продукт может легко быть модифицирован добавлением других типов моделей.

Программа курсовой работы полностью соответствует поставленному техническому заданию.
