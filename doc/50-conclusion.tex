\chapter*{Заключение}
\addcontentsline{toc}{chapter}{Заключение}

В ходе проделанной работы, был выполнен сравнительный анализ алгоритмов удаления невидимых линий и поверхностей и алгоритмов закраски.
Для разработки программы визуализации поверхностей, заданных функциями от двух переменных, была предложена оптимизация, заключающаяся в ускорении времени генерации изображения путём совмещения алгоритма Z-буфера и алгоритма закраски по Гуро.

Данный программный продукт может быть использован для демонстрации сложных графиков функций от двух переменных, их взаимного расположения.
Предоставлены возможности настройки визуальных характеристик поверхностей, а так же изменение ориентации и положения камеры с помощью манипуляторов клавиатура и мышь.

В исследовательском разделе расчётно-пояснительной записки выяснено, что время генерации изображения линейно зависит от количества полигонов на сцене, а максимальной производительности можно добиться при количестве потоков, равном числу логических ядер процессора.

Благодаря использованию объектно-ориентированного подхода при разработке, программный продукт может быть легко модифицирован добавлением других типов моделей.

Программа курсовой работы полностью соответствует поставленному техническому заданию.
